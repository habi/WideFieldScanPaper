\svnidlong
{$HeadURL$}
{$LastChangedDate$}
{$LastChangedRevision$}
{$LastChangedBy$}
\framebox{Author: \svnauthor; Rev: \svnrev; Last change: \svndate}% - URL: \url{\svnkw{HeadURL}}}
\section{Materials and Methods}
For each sample we obtained a varying amount of projections covering the desired FOV. These projection images have to be obtained in such a manner that they overlap each other slightly. This overlap is necessary for the compensation of variations in the imaging process and correct stitching of the single projections into one big projection image. 

After the acquisition of the subscan projection images, these image sets are corrected with the so called dark field and flat field images. The dark field images are obtained with no x-ray beam for the detection of intrinsic noise in the apparatus, the flat field images are obtained with x-ray beam, but without the sample. They are recorded to remove the varying beam profile brightness from the projection images.

After normalization, the projections of the single subscans are merged into one projection which overlaps the full FOV chosen by the end-user. To achieve a correct stitching of the subscan projections, the correct cutline to remove the overlap is calculated using the mutual difference between adjacent subscans. Since the amount of obtained projection varies for different positions of the sample in relation to the camera window --- at outer positions we record more projections compared to central positions --- we also have to interpolate projection images prior to stitching. All this has been achieved using a custom MATLAB\textsuperscript{\textregistered} program which incorporates the loading, normalizing, interpolation and correct stitching of the images into wide field projections.

After the merging of the normalized projections, the merged projections can then be reconstructed into the virtual tomography slices using a standard filtered backprojection algorithm or a FFT-based gridrec algorithm~\cite{Dowd2003} present at the beamline.
