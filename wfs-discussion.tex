% !TEX root = widefieldscan.tex
\svnidlong
{$HeadURL$}
{$LastChangedDate$}
{$LastChangedRevision$}
{$LastChangedBy$}
%
\section{Discussion}\label{sec:Discussion}

We present a method to laterally increase the field of view of tomographic imaging systems operated in parallel beam geometry and would like to call this method wide field synchrotron radiation based x-ray tomographic microscopy (WF-SRXTM). We defined scanning protocols for the optimization of the total imaging time versus the expected imaging quality, enabling a very fast acquisition of lower quality tomographic datasets, or acquisition of very high quality datasets in a longer time.

Even if the reduction in scanning time does introduce minor artifacts in the three-dimensional reconstruction, as shown in Figure~\ref{fig:BvsT}, an automated segmentation of the relevant features in the sample is still possible, even for protocols with greatly reduced scanning time. Additionally, reducing the scanning time reduces the radiation dose inflicted on the sample. For this publication, the radiation dose inflicted on the sample was of secondary concern, since our sample was neither radiation sensitive nor used for further investigations. %
\todo{Should we add reference to immunostaining experiments planned with Akira? $\rightarrow$ JCS} %
Nonetheless, such a reduction in radiation dose is a crucial step towards radiation sensitive experiments. Irradiating sensitive samples for a prolonged time is neither feasible nor desired, especially if biological development is studied, where multiple exposures of the same individual over a prolonged time-frame are required. With a suitable protocol the radiation dose can be reduced by \SI{84}{\percent}, which is a first step towards tomographic imaging of radiation sensitive samples and processes with ultra high resolution and enhanced field of view.

The field of view was increased three-fold by merging projections from three overlapping scans partially overlapping the sample and reconstructing these merged projections using the standard workflow at the TOMCAT beamline. As a consequence of the sampling theorem, an increased amount of projections had to be acquired, thus lengthening the acquisition time. To overcome this limitation, we defined multiple scanning protocols with a reduced amount of total projections and thus reduced acquisition time and delivered dose. All these protocols were evaluated for quality of the resulting reconstructions and compared to a gold standard scan. We have shown that the resulting quality can be simulated prior to scanning and thus provide a tool to choose a suited scanning protocol, based on the demands for scanning time optimization and quality of the resulting tomographic dataset.

Reducing the amount of projections for the central of the three subscans can be performed without loss of fidelity in the resulting reconstructions. Let us compare protocols D/E and H/I. For protocols E and I we acquired half the amount of projections for the central subscan $\textrm{s}_{2}$ than for protocols D and H. In both cases we reduce the scanning time by \SI{17}{\percent}, but keep the quality of the scan on nearly the same level (D: \SI{85}{\percent} vs. E: \SI{87}{\percent}, H: \SI{78}{\percent} vs. I: \SI{80}{\percent}).
% B--C = 13110/15732 = 0.8333, 100%--89%
% D--E = 10925/13110 = 0.8333, 85%--87%
% H--I = 8740/10488 = 0.8303; 78%--80%
We see that the interpolation of mission projections does not introduce errors in the resulting tomographic datasets and can thus perform tomographic scans using such protocols without loss of quality.

For protocols with an equal amount of total projections, but differing amount of projections for the individual subscans (C/D and M/N, marked in Figure~\ref{fig:NormalizedErrorPlot}) we observed minor differences in reconstruction quality. The qualities $E_{i_{norm}}$ of protocols C and D lie within their respective standard deviation (\SI{89}{\percent} vs. \SI{85}{\percent}), and the qualities of protocols M and N are comparable (\SI{69}{\percent} vs. \SI{67}{\percent}). Both protocols C and M are scanned without oversampling the central subscan, making interpolation necessary, while for protocols B and N we simply stitched together all the projection images. When deciding between two protocols with the same amount of total projections, it is thus desirable to favor the protocol where the central scan is not oversampled (\ie choosing protocol C instead of D). Even if this introduces additional computing time to interpolate projections prior to reconstruction, these protocols show an increased quality compared to protocols where the central scan is oversampled. Since an oversampling of the central scan does not add to the total reconstruction quality, this explanation seems natural. And since the outer parts of the sample contribute more to the total area of the projections, it naturally seems to be more sensible to choose a protocol where the sampling theorem is satisfied better for those parts of the sample. Both these findings are backed by our results.

With the defined protocols we open the possibility for the end-user to choose an acquisition mode suited to fulfill the constraints on beamtime and amount of samples to scan within the allotted beamtime. 

Additionally, two special use-cases for different protocols are worth mentioning. First, if the user desires to gain a very quick overview over his or her sample at a high resolution, \eg to quickly assess multiple samples in a short time, a time-saving protocol can be used. This is especially the case if for intricate sample preparations the integrity of the sample can only be judged with a tomographic scan. Multiple samples can thus be quickly scanned to gain an overview over which samples are worthwhile to scan at high resolution with high quality and which samples are to be discarded. It has to be mentioned that a quick overview could---in principle---be obtained with a low-resolution scan, which usually automatically accommodates a larger field of view. However, the resolution of such an overview scan is not always sufficient to detect interesting features in the samples which might be compromised.

The second special use-case worth mentioning is when the radiation dose inflicted on the sample has to be taken into consideration. If the end-user desires to have a tomographic scan at high resolution and large field of view, but strives to reduce the radiation dose inflicted on the sample, we now provide the possibility to perform such a scan during a reduced time in a semi-automated, objective way.

We have shown that the field of view of parallel beam tomographic end-stations can be increased up to five-fold and have three-dimensionally reconstructed multiple tomograms with a three-fold increase in field of view. The shown acquisition protocols are theoretically expandable for more than 5 subscans, although the reconstruction of wide field scans with 7 or more subscans would require an extremely powerful data processing infrastructure. The datasets shown in Figure~\ref{fig:BvsT} are binned scans resulting in datasets of 1024 slices, each with a size of 2792$\times$2792 pixels at a \SI{8}{\bit} depth. This amounts to a total size of the dataset of approximately \SI{7.5}{\giga\byte}. If we assume an un-binned scan with 7 overlapping subscans, the size of one stitched projection will be approximately 14000$\times$14000 pixels. The full dataset will consist of 2048 such slices, which add up to a total size for the full dataset of approximately \SI{383}{\giga\byte}. All datasets in this manuscript were reconstructed at \SI{8}{\bit}. Reconstructing this sample at \SI{16}{bit}---which is possible at TOMCAT---such wide field scan with a seven-fold increase in field of view would result in one single dataset with a size of \SI{0.75}{\tera\byte}.

Even if the amount of data to handle is huge, a wide field scan with a five-fold increase in field of view remains interesting, since it would enable the end-user to selectively reconstruct only regions of interest from large samples with ultra-high resolution.

Up to now, a two-step process was required to scan such regions from samples larger than the field of view. In a first step, a registered overview scan of the sample at lower resolution---and thus large field of view---was acquired. Then, using a custom-made software~\cite{Heinzer2008}, a region of interest was defined in this low-resolution dataset, and a high resolution local tomography scan---with small field of view---was performed for this marked region.

The extremely big datasets are a disadvantage when comparing WF-SRXTM with this two-step method. With the two-step method, only selective regions of interest, selected from a low-resolution tomogram, are scanned and reconstructed in high resolution, thus reducing the amount of data to process. But this two-step method is one of the big disadvantages of this method compared to WF-SRXTM, since it needs an extremely precise registration method to register the two datasets. Additionally, since for a WF-SRXTM scan all the data is recorded in a one-step process, the radiation dose is minimized. 

To overcome the limitation of the huge datasets, it also would be possible to integrate a method of only reconstructing partial regions of the full size datasets into the data processing pipeline of TOMCAT. After definition of a region of interest to be reconstructed out of the high-resolution wide field dataset, partial sinograms and partial reconstructions could be calculated, avoiding a process with two scans and thus reducing both radiation dose and size of the resulting dataset.