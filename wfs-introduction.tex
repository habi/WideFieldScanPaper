% !TEX root = widefieldscan.tex
\svnidlong
{$HeadURL$}
{$LastChangedDate$}
{$LastChangedRevision$}
{$LastChangedBy$}
%
\section{Introduction}
\subsection{What's the problem?\todo{FOV too small for acinus!}}
In our group, we are studying the structural alterations of the lung parenchyma throughout lung development \cite{Schittny2008,Mund2008,Tsuda2008}. Of particular interest is the development of the functional respiratory lung unit, the so-called acinus. The acinus is defined as the complex of alveolated airways distal to the terminal bronchiole \cite{Rodriguez1987}, in which the gas-exchange in the lung takes place. We are interested in obtaining information about the branching pattern of the airways of the acinus and the airflow within and thus need the three-dimensional structural information only available with datasets provided by tomography methods.

Since the defining parts of the acinus, the tissue septa that form the gas-exchanging alveoli inside the lung are around \SI{5}{\micro\meter} thin \cite{Weibel2009} we need to obtain high resolution images of the terminal airways to be able to resolve and study the tissue in this region of the lung. In addition, the acinus is growing in size over the course of the postnatal lung development, thus we also need the tomographic dataset to caontain a large field of view to be able to visualize and examine a full acinus.

The field of view of microscopy-based imaging methods is limited by the camera and optics, a large field of view resulting in a large sample volume can only be acquired with lower magnification and vice-versa. Lab-based micro-computed tomography stations (\micro CT) offer a field of view suited to study entire acini, but the resolution at this magnifications is too low to resolve all alveolar septa. Albeit \micro CT are catching up on the resolution side, high resolution synchrotron radiation based x-ray tomographic microscopy (SRXTM) beamlines provides the needed high resolution combined with an unmatched image quality by having a very high flux and monochromatic beam. Nonetheless, at the TOMCAT beamline \cite{Stampanoni2007} at the Swiss Light Source, Paul Scherrer Institute, Villigen, Switzerland, the field of view at the 10$\times$ magnification resulting in \SI{0.74}{\micro\meter} voxel size needed to securely visualize all alveolar septa is limited to 1.52$\times$\SI{1.52}{\milli\meter}.

\subsection{What's the goal?\todo{Increase the FOV, keep the high resolution}}
Increasing the field of view along the rotation axis of the sample can easily be achieved by stacking multiple tomographic datasets on top of each other, but this is only a feasible method for long and thin samples. 

Increasing the field of view in the direction perpendicular to the rotation axis of the sample cannot easily be achieved by placing tomographic datasets next to each other. For a small detector, it is necessary to merge several projection images overlapping the desired field of view prior to reconstructing the tomographic dataset. Increasing the field of view also makes it necessary to acquire more projections. This is a consequence of the restrictions of the sampling theorem, which states that for a detector with D, we need to acquire $\textrm{P}=\textrm{D}\frac{\pi}{2}$ projections \cite[page 186]{Kak2002}. This leads to an increase of the acquisition time for larger fields of view.

\subsection{How did we do it?\todo{Merging images, optimizing protocols}}
To overcome the mentioned problems and limitations, we developed a scanning protocol which enables us to merge several independently acquired sets of projections to overlap the desired field of view. Additionally, through optimization of the parameters of the projection sets we established several different protocols to obtain high quality tomographic datasets within a reduced time compared to a non-optimized scan.

\subsection{What's cool about this?\todo{Large FOV, small pixels! Radiation dose is reduced}}
An increase in the sample volume contained in a tomographic dataset with small pixel size permits an unrestricted high-resolution three-dimensional view inside the terminal airways over the postnatal lung development only limited by the sample size. Reducing the total acquisition time of a tomographic scan also reduced the radiation dose inflicted on the sample while keeping the quality of the resulting three-dimensional dataset on a level comparable to an un-optimized scan. Reducing the radiation dose on the sample is a first step on the way to performing in vivo SRXTM at the TOMCAT beamline.