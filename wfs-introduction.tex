% !TEX root = widefieldscan.tex
\svnidlong
{$HeadURL$}
{$LastChangedDate$}
{$LastChangedRevision$}
{$LastChangedBy$}
%
\section{Introduction}

The functional respiratory lung unit---the so-called acinus---is defined as the complex of alveolated airways distal of a last purely conducting airway, the terminal bronchiole \cite{Rodriguez1987}. The total of all acini forms the lung parenchyma, the area where the pulmonary gas-exchange takes place. While the structural development of the gas-exchange region including the alveolar septa is quite well characterized \cite{Schittny2007a,Schittny2008,Mund2008}, the development of the three-dimensional structure of its functional unit---of the acini---was not much studied due to the lack of suitable methods.

It is our goal to study the branching pattern of the acinar airways as well as the airflow within it. Tomographic methods, in particular synchrotron radiation based tomographic microscopy can access this kind of information nondestructively and noninvasively.

In order to visualize the thin sheets of tissue (alveolar septa) forming the gas-exchanging alveoli, a resolution in the order of one micron is required. An entire acinus is usually larger than the field of view of the tomographic microscope \cite{Rodriguez1987,Weibel2009}, being the latest limited by the chosen optical configuration. Usually, a large field of view resulting in a large sample volume can only be acquired with low magnification and vice-versa. Lab-based micro-computed tomography stations (\micro CT) could potentially be used to study acini, but the resolution of such systems is too low to resolve all alveolar septa. Even if \micro CT stations are catching up, synchrotron radiation based tomographic microscopy beamlines provide the necessary high resolution combined with unmatched image quality.

Up to now, the price to pay for this high resolution was a limited field of view. For instance at the TOMCAT beamline \cite{Stampanoni2007} at the Swiss Light Source, Paul Scherrer Institute, Villigen, Switzerland, the field of view at a 10$\times$ magnification (\SI{0.74}{\micro\meter} voxel size) is limited to 1.52$\times$\SI{1.52}{\milli\meter}, insufficient for the imaging of an entire acini at high resolution.

Increasing the field of view perpendicular to the rotation axis of the sample cannot easily be achieved by placing tomographic datasets next to each other. It is instead necessary to merge several projections overlapping the desired field of view prior to tomographic reconstruction. Obviously, to satisfy the sampling theorem, increasing the field of view also requires to acquire more projections, finally resulting in an increased acquisition time.

We developed such a method to merge several independently acquired sets of projections to increase the field of view of the resulting tomographic dataset. In addition, by optimization of the number of recorded projections, we established different scanning protocols with a user-defined balance between acquisition time and image quality.

Because the total acquisition time is directly linked to the radiation imparted to the sample, it is obvious that such protocols also affect radiation damage and constitute an important optimization tool for radiation sensitive experiments.