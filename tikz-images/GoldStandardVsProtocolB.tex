%\documentclass{article}
%\usepackage[demo]{graphicx}
%\usepackage{subfig}
%\usepackage{tikz}
%\usepackage{multirow}
%\usepackage{siunitx}
%\begin{document}
%\begin{figure}
%\centering
%%%%%%%%%%%%%%%%%%%%%%%%%%%%%
\def\scale{0.618}
\subfloat[Desired FOV]{%
	\label{subfig:fovneed}%
	\begin{tikzpicture}[scale=\scale]
	%	\draw [dashed] (-1,-1) grid (7,7);
		\draw [fill=gray!25] (0,0) rectangle (6,6);
		\fill [semitransparent] (3,3) circle (3);
		\draw (3,3) circle (3);
		\draw [white,ultra thick,<->] (0,3) -- node [above] {3072 px} (6,3);
	%	\draw [step=2] (0,0) grid (6,6);
	\end{tikzpicture}%
}\hfill
\subfloat[Gold Standard; covering the FOV with 9 independently reconstructed small scans.]{%
	\label{subfig:goldstandard}%
	\begin{tikzpicture}[scale=\scale]
	%	\draw [dashed] (-1,-1) grid (7,7);
		\fill [color=gray!25] (0,0) rectangle (6,6);
		\foreach \x in {1,2,3}
			\foreach \y in {1,2,3}
				\draw [fill=gray] (2*\x-1,2*\y-1) circle (1) node {\x,\y};
		\fill [semitransparent] (3,3) circle (3);
		\draw (3,3) circle (3);
		\draw [step=2] (0,0) grid (6,6);
		\draw [white,ultra thick,<->] (2,3.5) -- node [above] {1024 px} (4,3.5);
	\end{tikzpicture}%
}\hfill
\subfloat[Gold Standard; covering the FOV with merged projections from one central and two ring scans.]{%
	\label{subfig:protocolb}%
	\begin{tikzpicture}[scale=\scale]
%		\draw [dashed] (-1,-1) grid (7,7);
		\draw [fill=gray!25] (0,0) rectangle (6,6);
		\fill [semitransparent] (3,3) circle (3);
		\draw (3,3) circle (3);
		\draw (0,3) -- (2,3);
		\draw (4,3) -- (6,3);
		\draw (3,3) circle (1);
		\def\shift{0.3}
		\node at (3,5-\shift) {ring scan 1};
		\node at (3,1+\shift) {ring scan 2};
		\node at (3,3+\shift) {central};
		\node at (3,3-\shift) {scan};
		\def\angle{155}
		\draw [white,ultra thick,<->] (3,3) +(\angle:1) -- node [sloped,midway,above] {1024 px} +(\angle:3); 
%		\draw [red,ultra thick,<->] (0,3) -- node [above] {1024 px} (-10:110);
	\end{tikzpicture}%
}
%%%%%%%%%%%%%%%%%%%%%%%%%%%%%	
%\caption{Projection Setup}
%\end{figure}
%\end{document}