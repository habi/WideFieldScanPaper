%!TEX root = widefieldscan.tex
\svnidlong
{$HeadURL$}
{$LastChangedDate$}
{$LastChangedRevision$}
{$LastChangedBy$}

\begin{center}
	\fbox{
		\begin{minipage}{.618\textwidth}
		The file of this section is \url{\svnkw{HeadURL}}, was last changed at: \svnfileday.\svnfilemonth.\svnfileyear \space \svnfilehour:\svnfileminute \space (UTC\svnfiletimezone) and is at revision \svnkw{LastChangedRevision}.
		\end{minipage}
	} 
\end{center}

\section{Summary}
The proposed method to increase the FOV of SRXTM establishes to increase the diameter of the visible FOV up to five times what is conventionally available. We have shown that we can provide different scanning protocols based on needs of the end-user of TOMCAT. The end-user has the possibility to choose a suitable scanning protocol depending on a balance between acquisition time and reconstruction quality. We managed to reduce the image acquisition time of one sample down to \SI{14}{\percent} of the comparable gold standard scan of the same sample while keeping the quality of the reconstructed tomographic dataset on a level that still permitted automated segmentation of the lung structure and surrounding airspace.

Further increases of the FOV are only theoretically limited, but face problems with the amount of data to process. The dataset shown in figure~\ref{fig:LungSlabSophie} is composed of 1024 tiff-Files each with a size of 4852$\times$4852 pixels, which adds up to a total size of nearly \SI{23}{\giga B}. Calculations and visualizations of datasets of this size require great optimizations of the processing queue and automation of calculations\todo{elaborate a bit more\ldots}.

The stitching of the multiple overlapping has been automated using custom made MATLAB scripts and will be implemented at the beamline for end-user access.

\section{Outlook}
\begin{itemize}
	\item Implementation @ TOMCAT
	\item needed for Skeletonization of entire acini $\rightarrow$ follow-up paper
\end{itemize}