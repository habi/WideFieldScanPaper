% !TEX root = widefieldscan.tex
\svnidlong
{$HeadURL$}
{$LastChangedDate$}
{$LastChangedRevision$}
{$LastChangedBy$}
%
\section{Summary}\label{summary}

A method to increase the lateral field of view of tomographic imaging has been established. The method enables the high-resolution tomographic imaging of large samples---wider than the field of view of the imaging setup---in multiple steps, combined into an semi-automated tool chain. The enhanced field of view makes it possible to visualize entire acini inside rat lung samples, something which is not easily achieved using the conventional tomographic setup present at TOMCAT. With this newly developed WF-SRXTM-method we have been able to acquire multiple datasets of the functional lung unit in the gas-exchange region of the terminal lung unrestricted by the field of view of the TOMCAT beamline.

We have shown that an increase in the field of view of up to five times is feasible. Further increases of the field of view are only theoretically limited, but face problems with the amount of data to process. 

Different optimized scanning protocols for covering a large field of view have been validated and are now provided for the end-users of the TOMCAT beamline. End-user now have the possibility to choose suitable scanning protocols depending on a balance between acquisition time and expected reconstruction quality. Depending on this balance, we manage to reduce the image acquisition time by \SI{84}{\percent} of the comparable gold standard scan, while keeping the quality of the reconstructed tomographic dataset on a level still permitting automated segmentation of the lung structure and surrounding airspace, as shown in section~\ref{subsec:comparison}. The reduction in acquisition time obviously reduces the time during which the sample is irradiated by the synchrotron radiation and thus reduces the radiation dose inflicted on the sample.